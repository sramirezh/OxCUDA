% Common header definitions and settings for use in the other documentation.


% Packages
\usepackage[T1]{fontenc}    % Font Encoding
\usepackage{utopia}         % Font
%\usepackage{courier}        % Monospaced font % This messes up the code listings, as it is wider than the default.
\usepackage{titling}        % Make modifications to \maketitle
\usepackage{fancybox}       % Boxes around terminal listings.
\usepackage{fancyvrb}       % Code listings, terminal sessions, etc.
\usepackage{color}          % Define and use colours
\usepackage{tocloft}        % TOC modification
\usepackage{datetime}       % Date formatting
\usepackage{float}          % Define new float types (code)
\usepackage{caption}        % Modify caption appearance
\usepackage{todonotes}      % Add todo notes in the margin
\usepackage{framed}         % shaded environment provides background colours
\usepackage{fourier-orns}   % Ornaments around title
\usepackage{siunitx}        % Align on decimal point column type
\usepackage{ifdraft}        % Test for draft mode
\usepackage{tikz}           % Drawing Trees
\usepackage{pdflscape}      % provides landscape environment for pdflatex.
\usepackage{ifthen}         % Provides conditional tests
\usepackage{hyperref}       % Links, and nameref command




% Define some commannds for displaying the names of different things
\newcommand{\filename}[1]{\texttt{#1}}
\newcommand{\command}[1]{\texttt{#1}}
\newcommand{\var}[1]{$#1$}
\newcommand{\mathvar}[1]{#1}
\newcommand{\codefragment}[1]{\texttt{#1}}
\newcommand{\confsnippet}[1]{\texttt{#1}}

\newcommand{\class}[1]{\texttt{#1}}
\newcommand{\func}[1]{\texttt{#1}}



% Set up the appearance of todo notes
\newcommand{\todonote}[1]{\todo[backgroundcolor=white, linecolor=black, 
                                size=\small, noline]{#1}}



% Environment for showing bits of code, terminal listings, and so on.
\definecolor{box-grey}{gray}{0.4}
\newcommand{\codelabel}[1]{\textcolor{black}{#1}}
\DefineVerbatimEnvironment
    {Code}
    {Verbatim}
    {frame=single, fontsize=\scriptsize, framesep=3mm, formatcom=\vspace{7pt}, 
      xleftmargin=5mm, xrightmargin=5mm, rulecolor=\color{box-grey},
      labelposition=topline, numbersep=8pt, baselinestretch=1.2,
      numbers=left, samepage=true}
% N.B. can use option label=\codelabel{??} as needed.



% Wrapper for the above for teminal listings
% This is a pretty nasty hack, but I'm really tired. ;)
\definecolor{shadecolor}{gray}{0.95}
\setlength{\FrameSep}{-5mm}
\newenvironment{Term}
    {\begin{center}\begin{Sbox}\begin{minipage}{\linewidth-17.4mm}}
    {\end{minipage}\end{Sbox}{\setlength{\fboxsep}{3mm}\vspace*{8pt}\fcolorbox{box-grey}{shadecolor}{\TheSbox}\vspace*{6pt}}\end{center}}

\DefineVerbatimEnvironment
    {CodeTerm}
    {Verbatim}
    {frame=none, fontsize=\scriptsize, framesep=3mm, formatcom={}, 
      xleftmargin=0mm, xrightmargin=0mm, rulecolor=\color{box-grey},
      labelposition=topline, numbersep=8pt, baselinestretch=1.2,
      numbers=left, samepage=true,
      numbers=none}

% Usage:
%\begin{Code}
%\begin{CodeTerm}
%Hello, World.
%\end{CodeTerm}
%\end{Term}


% Rename the abstract
\renewcommand{\abstractname}{Introduction}



% Remove section numbering
\setcounter{secnumdepth}{0}



% Remove bold and add dots to sections in the TOC
\renewcommand\cftsecfont{\normalfont}
\renewcommand\cftsecpagefont{\normalfont}
\renewcommand{\cftsecleader}{\cftdotfill{\cftsecdotsep}}
\renewcommand\cftsecdotsep{\cftdot}
\renewcommand\cftsubsecdotsep{\cftdot}
% Show only sections in the TOC
\setcounter{tocdepth}{1}


% Configure the \maketitle command a little
\setlength{\droptitle}{-30pt}
\predate{\begin{center}\small}
\postdate{\par\end{center}}



% Set up the desired date format (20^th July 2011)
% N.B. this is very similar to the default with the 'nodayofweek' option.
\newdateformat{mydate}{\ordinaldate{\THEDAY} \monthname[\THEMONTH] \THEYEAR}
\mydate



% New float type for code listings
\floatstyle{plain}
\floatname{listing}{Listing}
\newfloat{listing}{p}{lol}



% Set up appearance of captions
\captionsetup{margin=30pt,font=small,labelfont=bf}



% Remove ugly boxes etc around hyperlinks.
\hypersetup{pdfborder = {0 0 0}}




% Tree Drawing
% {A, B, {C, D}, {E, F}}
\newcommand{\treeDrawABCDEF}{%
    \[\begin{tikzpicture}
        \node{$\{A, B\}$}
            child{node {$\{C, D\}$}}
            child{node {$\{E, F\}$}}
        ;
    \end{tikzpicture}\]
}




% Add a command to produce nice looking tildes: \urltilde
% http://stackoverflow.com/questions/718479/what-is-the-best-way-to-produce-a-tilde-in-latex-for-a-website/3900556#3900556
\catcode`~=11 % make LaTeX treat tilde (~) like a normal character
\newcommand{\urltilde}{\kern -.15em\lower .7ex\hbox{~}\kern .04em}
\catcode`~=13 % revert back to treating tilde (~) as an active character




% Allow document title to be set from main file.
\newcommand{\thedocumenttitle}{}
\newcommand{\documenttitle}[1]{%
    \renewcommand{\thedocumenttitle}{#1}
}



% Title Info
\title{Flamingo Auto-Tuning\\\decofourleft~~~\thedocumenttitle~~~\decofourright}
\author{Ben Spencer\\\normalsize\href{mailto:ben@mistymountain.co.uk}{ben@mistymountain.co.uk}}
\date{Last updated: \today}



